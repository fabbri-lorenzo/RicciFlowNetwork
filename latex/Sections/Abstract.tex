\begin{abstract}
    This project uses Ricci Flow as a geometric approach to detect community structures in networks. Specifically, it applies Ollivier-Ricci curvature to adjust edge weights in a graph, iterating the process to shrink intra-community edges and stretch inter-community edges. Then, surgery is performed to separate the graph into distinct connected components, representing communities.

    After having tested the Ricci Flow method on synthetic graphs, we applied it to a real dataset: Zachary's Karate Club graph~\cite{ZacharyKarateClubGraph}, aiming to accurately identify its pre-labelled communities.
\end{abstract}