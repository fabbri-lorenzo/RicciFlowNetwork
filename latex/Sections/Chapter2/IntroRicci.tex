%**************** INTRODUCTION TO OLLIVIER RICCI CHAPTER ******************
Initially introduced by Richard S.~Hamilton in the early 1980s, Ricci Flow emerged as a powerful method in differential geometry. At its core, it seeks to ``smooth out'' geometric irregularities of manifolds by evolving the underlying Riemannian metric through a partial differential equation similar to the classical heat equation. Recall that a manifold is a topological space that locally resembles a Euclidean space, and a \emph{Riemannian manifold} is such a space equipped with an inner product on each tangent space, allowing for measurements of angles, distances, and curvature. 

Curvature, in particular, is fundamental to geometry, as it describes how space bends or deviates from flatness. For example, on a two-dimensional surface embedded in a three-dimensional space, curvature can be visualized by examining the deviation of geodesics (the generalization of ``straight lines'' in curved spaces) from parallelism, or by looking at how areas and angles are distorted compared to those in flat Euclidean geometry. Extending to higher dimensions and more abstract manifolds requires careful definitions but retains this key notion of ``spatial bending.'' 

