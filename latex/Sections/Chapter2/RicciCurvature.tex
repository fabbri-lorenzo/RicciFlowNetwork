\section{Riemannian Geometry and Curvature Notions}
In classical Riemannian geometry, curvature can be examined from multiple perspectives. One can study the \emph{sectional curvature}, which measures how a two-dimensional plane (spanned by two tangent directions) curves. One can also investigate the \emph{Ricci curvature}, which is obtained by summing or averaging the sectional curvatures over all planes containing a given direction. Ricci curvature has special importance: for instance, it appears in Einstein’s equations of General Relativity, linking geometry to matter and energy distributions in spacetime. 

Concretely, let $(M,g)$ be a Riemannian manifold, where $M$ is a smooth manifold and $g$ is the metric tensor. The Ricci curvature $\mathrm{Ric}$ is derived as a contraction of the Riemann curvature tensor, itself an operator capturing how much nearby geodesics converge or diverge. Positive Ricci curvature typically implies that geodesics tend to converge, reflecting a ``crowded'' or positively curved geometry akin to the sphere. Negative Ricci curvature implies geodesics tend to diverge, mirroring a hyperbolic or “saddle-like” structure. Zero Ricci curvature is the hallmark of Ricci-flat manifolds, with many implications for geometry and topology.

\subsection{Hamilton's Ricci Flow Equation}
Hamilton introduced the Ricci Flow as the PDE:
\[
\frac{\partial g_{ij}}{\partial t} = -2 \, R_{ij},
\]
where $g_{ij}$ are the components of the metric tensor $g$ in local coordinates and $R_{ij}$ are the components of the Ricci curvature tensor. Informally, each infinitesimal piece of the manifold changes in time, guided by curvature. Regions of \emph{high positive} Ricci curvature shrink faster, while regions of \emph{negative} Ricci curvature expand. This leads to a \emph{flow} that tends to smooth out the geometric and topological features of $M$. 

One of the most famous applications of Ricci Flow on manifolds is Grigori Perelman's resolution of the Poincar\'{e} Conjecture and the more general Geometrization Conjecture for three-dimensional manifolds. Perelman’s work introduced the notion of \emph{Ricci Flow with surgery}, a procedure to remove singular regions (places where curvature blows up to infinity) and continue the flow on the remaining parts. In 3D manifolds, these singularities can be visualized as ``neck pinches'' that effectively separate the manifold into topologically simpler pieces. Perelman showed that by performing a series of well-defined surgeries, one could decompose a three-dimensional manifold into model geometric pieces, completing Hamilton’s program toward a proof of the Geometrization Conjecture.

\subsection{From Smooth Settings to Discrete Geometry}
While Ricci Flow is classically defined on smooth manifolds, there has been considerable interest in transferring these ideas to \emph{discrete} or combinatorial settings such as polyhedral surfaces, graphs, and complex networks. The general question is how to define concepts like ``curvature'' when one does not have a smooth manifold or a Riemannian metric in the usual sense. Instead, discrete analogs focus on adjacency, distances along edges, and combinatorial properties that mimic or reflect continuum notions.

For surfaces composed of polygons (triangulations), one can define the curvature at a vertex via angle deficits, a concept dating back to classical differential geometry of polyhedral surfaces. However, for higher-dimensional graphs and networks that are not neatly embedded in any Euclidean space, a more general curvature definition is needed—one that depends mostly on the underlying distances and probability measures rather than an explicit embedding. 