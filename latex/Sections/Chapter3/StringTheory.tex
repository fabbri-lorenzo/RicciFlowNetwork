%**************** NON-LINEAR SIGMA MODELS AND STRINGS ******************
\section{Non-Linear Sigma Models and String Theory}
The standard starting point of string theory is \emph{Polyakov action}, which represents a bosonic, classical, one-dimensional, string~\cite{weigand:string,polchinski:strings}. This latter traces out a two-dimensional worldsheet $\Sigma$, whose intrinsic coordinates are $\xi^a = (\tau, \sigma)$, $a = 1,2$. Further, the string is embedded in a $26$-dimensional spacetime, and from this point of view it's described by the bosonic coordinates $X^\mu(\xi)$, $\mu = 0, \dots 25$. We focus on the \emph{closed string}, defined by the periodicity condition $X^\mu(\tau,\sigma) = X^\mu (\tau, \sigma + l)$.

The metric on spacetime is denoted by $g_{\mu\nu}$, while the metric on the worldsheet is $\gamma_{ab}$. For a flat spacetime, with Minkowski metric $g_{\mu\nu} \equiv \eta_{\mu\nu} = \textup{diag}(-1,+1,+1,+1)$, the action reads
\begin{equation}\label{eq:polyakov}
    S_P [X^\mu(\xi), \gamma_{ab}(\xi)] = -\frac{1}{4\pi\alpha'} \int_\Sigma \ud \tau \ud \sigma \sqrt{-\det(\gamma)} \, \gamma^{ab} \de_a X_\mu (\xi) \de_b X^\mu (\xi),
\end{equation}
where $\alpha'$ is a characteristic parameter of the string, related to the string tension by $T = 1 / 2\pi\alpha'$.

The symmetries of this action allow us to fix the gauge such that the worldsheet metric is flat, $\gamma_{ab} = \eta_{ab}$. Reintroducing the spacetime metric $g_{\mu\nu}$ for completeness, we obtain
\begin{equation}\label{eq:polyakov-metric}
    S_P = - T \int_\Sigma \ud^2 \xi \, g_{\mu\nu}(X) \de_a X^\mu \de^a X^\nu.
\end{equation}

Basically, this action represents a $2$-dimensional field theory on the worldsheet, where the coordinates $X^\mu$ are 26 dynamical fields. This allows us to quantize the theory with the usual quantization prescription, based on the substitution of the classical Poisson brackets defined on a symplectic manifold with the commutators of operators acting on a Hilbert space.

After quantization, the massless spectrum of the closed string contains a \emph{graviton} $\gamma_{\mu\nu}$, which resembles a gravitational wave at low energies, a scalar field $\phi$ called \emph{dilaton} and an antisymmetric two-tensor $b_{\mu\nu}$ called \emph{Kalb-Ramond tensor}.

Because of the presence of the graviton, one could wonder what happens for a non-flat spacetime. To understand it, it's first necessary to redefine the coordinates as a constant $X^\mu_0$ plus some other arbitrary fields $Y^\mu$,
\begin{equation}
    X^\mu (\xi) = X^\mu_0 (\xi) + \sqrt{\alpha'} \, Y^\mu (\xi).
\end{equation}

The term in the Lagrangian can be expanded as
\begin{equation}\label{eq:expansion-coordinates}
\begin{split}
    &g_{\mu\nu}(X) \de_a X^\mu \de^a X^\nu \\
    &= \alpha' \Bigl[ g_{\mu\nu}(X_0) + \sqrt{\alpha'} g_{\mu\nu,\rho}(X_0)Y^\rho(\xi) \\ &+ \frac{\alpha'}{2} g_{\mu\nu,\rho\sigma}(X_0)Y^\rho(\xi Y^\sigma(\xi)) + \dots \Bigr] \de_a Y^\mu \de^a Y^\nu ,
\end{split}
\end{equation}
where $g_{\mu\nu,\rho} \equiv \de_\rho g_{\mu\nu}$. We obtained an expansion in $\alpha'$, where each term is an interaction term for the fields $Y^\mu$, with couplings given by the derivatives of the metric. 

One additional feature of the Polyakov action~\eqref{eq:polyakov} is the invariance under \emph{conformal transformations}. Those are diffeomorphisms on a Riemannian manifold which preserve the metric up to rescaling, that is,
\begin{equation}\label{eq:conformal-transformation}
    g(x) \to \tilde{g}(\tilde{x}) = e^{2\omega(\tilde{x})} g(\tilde{x}).
\end{equation}

Without going into the details, the presence of this symmetry is considered as a consistency condition for the theory, as it allows for a perturbative interpretation of the interactions. 

Since the action which contains~\eqref{eq:expansion-coordinates} describes an interacting quantum field theory, it must undergo renormalization, to cure divergences. A generic property of renormalized theories is that the coupling constant is running. This is, however, incompatible with the conformal symmetry~\eqref{eq:conformal-transformation}, it implies scale-invariance.

Therefore, to tackle the quantum anomaly of the conformal symmetry, one should impose the \emph{$\beta$-function} vanishes,
\begin{equation}
    \beta(g_{\mu\nu}) = M \frac{\de}{\de M} g_{\mu\nu} \overset{!}{=} 0.
\end{equation}

A similar argument can be pursued for the other two particles in the closed string spectrum, the dilaton $\phi$ and the Kalb-Ramond form $b_{\mu\nu}$. First, the generalization of the action~\eqref{eq:polyakov-metric} which includes those fields is
\begin{equation}
    S_\sigma = -\frac{T}{2} \int_\Sigma \ud^2 \xi \sqrt{-\det(\gamma)}\left[ \left( \gamma^{ab} g_{\mu\nu}(X) + i \epsilon^{ab} b_{\mu\nu}(X) \right) \de_a X^\mu \de_b X^\nu + \alpha' \mathcal{R} \phi(X) \right],
\end{equation}
where $\epsilon^{ab}$ is the $2$d Levi-Civita symbol, and $\mathcal{R} = \mathcal{R}(\gamma)$ is the Ricci scalar on the worldsheet.

Defining the field strength $H_{\mu\nu\rho} = \de_\mu b_{\nu\rho} + \de_\nu b_{\rho\mu} + \de_\rho b_{\mu\nu}$, the vanishing of the beta functions reads
\begin{subequations}\label{eq:beta-functions}
\begin{align}
    \beta(g_{\mu\nu}) &= \alpha' \left( R_{\mu\nu} - \frac{1}{4} H_{\mu\lambda\rho} H^{\mu\lambda\rho} + 2 \cov_\mu \cov_\nu \phi \right) + O(\alpha'^2)\overset{!}{=} 0,\\
    \beta(b_{\mu\nu}) &= \alpha' \left( \frac{1}{2} \cov^\rho H_{\rho\mu\nu} + \cov^\rho \phi H_{\rho\mu\nu} \right) + O(\alpha'^2)\overset{!}{=} 0, \\
    \beta(\phi) &= \alpha' \left( \frac{1}{2} \cov_\mu \phi \cov^\mu \phi - \frac{1}{2} \cov^2 \phi - \frac{1}{24} H_{\mu\nu\rho} H^{\mu\nu\rho} \right) + O(\alpha'^2)\overset{!}{=} 0.
\end{align}
\end{subequations}

The above equations are constraints for the spacetime fields $(g,b,\phi)$, imposed to preserve conformal invariance of the quantum string. However, since those fields should be dynamical on spacetime, those must also be their equations of motion. This leads to the following \emph{low-energy effective action}, which has~\eqref{eq:beta-functions} as equations of motion
\begin{equation}
    S_{26} = \frac{1}{k^2_0} \int \ud^{26}x \sqrt{\det(g)} e^{-2\phi} \left( \mathcal{R}(g) - \frac{1}{12} H_{\mu\nu\rho} H^{\mu\nu\rho} + 4 \cov_\mu \phi \cov^\mu \phi \right).
\end{equation}

To set this problem to a more general ground, and understand how this model is related to Ricci Flow, let's define more accurately what a $\sigma$-model is in field theory. A $\sigma$-model is a field theory for a field $\Phi \colon \Sigma \to \M$ that takes values in a manifold $\M$. Traditionally, $\Sigma$ is the spacetime on which the field theory lives, and $\M$ is called the target space. If the target space carries some linear structure, like a vector space, then the whole physical system is called a linear $\sigma$-model. For general manifolds such as generic Riemannian ones, it is then called a non-linear $\sigma$-model. What we did above is to study the $\sigma$-model spacetime renormalization effects on the target space $\M$, and this is indeed an instance of Ricci Flow~\cite{ricci-flow-renormalization-group}.