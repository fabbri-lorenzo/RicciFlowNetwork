%**************** INTRODUCTION TO CODE CHAPTER ******************
The objectives of the developed code were
\begin{itemize}
    \item \emph{Implementation of a Ricci Flow method} able to evaluate the Ollivier-Ricci curvature of a given graph and update its edges' weights accordingly. Then we wanted the method to perform surgery on weakly connected edges to allow for community detection.
    
    For these purposes we relied on Ollivier-Ricci library developed by Ni et al. \cite{Ollivier-RicciLib}, see \autoref{sec5.1}.
    
    \item \emph{Plotting graphs and communities} to appreciate the behaviour of the method and obtain graphical results.
    
    \item \emph{Testing the method} on synthetic graphs, trying to benchmark with an analogue test made by Ni et al. ~\cite{Ni:communitydetectionnetworksricci}. Results and further details are given in \autoref{sec5.2}.
    
    \item \emph{Evaluation of performances} by comparing the method with other commonly used community detection methods.  
    
    \item \emph{Application on real world data}. To do this we chose Zachary's Karate Club graph, which is directly accessible from Networkx library. Results and further details are given in \autoref{sec5.3}.
    
\end{itemize}

In this chapter we present the main ideas and tools related to the code. To get more insights on how the code as been constructed and subdivided (i.e., various classes and functions) we recommend consulting \textit{\href{https://fabbri-lorenzo.github.io/RicciFlowNetwork/}{CodeDocumentation}} or the GitHub of the whole project: \textit{\href{https://github.com/fabbri-lorenzo/RicciFlowNetwork}{RicciFlowNetwork}}.