\section{Optimal Transport and Ollivier's Ricci Curvature}
A major breakthrough in defining a curvature notion for general metric spaces (including discrete networks) came via \emph{optimal transport}. Historically, the optimal transport problem, originating in the work of Gaspard Monge in the 18th century, asks how to map one mass distribution into another with minimal transportation cost. Subsequent reformulations by Leonid Kantorovich turned this into a linear optimization problem known as the \emph{Kantorovich relaxation}. 

If $(X, d)$ is a metric space, and we have two probability measures $\mu$ and $\nu$ on $X$, the \emph{Wasserstein distance} (also called the earth mover’s distance) measures how much ``effort'' is needed to move mass from $\mu$ to $\nu$, given the metric $d$. Specifically, one solves an optimization problem that tries to minimize the total cost of moving infinitesimal amounts of mass from one location to another. 

Yann Ollivier harnessed this framework to define a notion of \emph{coarse Ricci curvature} on general metric measure spaces. Ollivier’s definition, now widely called \emph{Ollivier Ricci curvature}, is based on considering small probability balls of radius $\varepsilon$ around points (or sometimes discrete probability measures concentrated on nearest neighbors in a graph) and calculating how much these small balls cost to move one onto the other under optimal transport. If moving these balls requires comparatively more effort than just their pairwise distance would suggest, the edge or connection between them is deemed negatively curved; if it requires less effort, it is positively curved. This emerges from an analog of the well-known statement in Riemannian geometry that Ricci curvature controls how geodesic balls deviate from each other, which can be recast in terms of mass transport.
