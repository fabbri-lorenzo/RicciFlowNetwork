%**************** INTRODUCTION TO OLLIVIER RICCI CHAPTER ******************
Ricci Flow arose historically as a powerful method in differential geometry, initially introduced by Richard S.~Hamilton in the early 1980s. At its heart, Ricci Flow seeks to ``smooth out'' geometric irregularities of manifolds by evolving the underlying Riemannian metric through a partial differential equation (PDE) reminiscent of the classical heat equation. In the context of differential geometry, a manifold is a topological space that locally resembles Euclidean space, and a \emph{Riemannian manifold} is such a space equipped with an inner product on each tangent space, making it possible to measure angles, distances, and curvature. 

Curvature, in particular, is fundamental to geometry: it describes how space bends or deviates from flatness. On a two-dimensional surface embedded in three-dimensional space, for example, curvature can be visualized by examining the deviation of geodesics (the generalization of ``straight lines'' in curved spaces) from parallelism, or by looking at how areas or angles are distorted compared to those in flat Euclidean geometry. The extension to higher dimensions and more abstract manifolds involves careful definitions but retains this key notion of ``spatial bending.'' 


