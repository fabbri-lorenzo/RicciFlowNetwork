\section{Discrete Ricci Flow}
Following Ollivier’s curvature definition, we can define a discrete analog of Ricci Flow by iteratively adjusting the weights of edges in $G$ according to the local curvature. In classical geometry, Ricci Flow modifies the metric in a continuous manner according to $-\,2\,\mathrm{Ric}$. In the discrete setting, one can proceed iteratively:
\[
w_{xy}^{(i+1)} \;=\; w_{xy}^{(i)} \;-\; \eta \,\kappa_{xy}^{(i)} \; d_{xy}^{(i)},
\]
where $w_{xy}^{(i)}$ is the weight of the edge $(x,y)$ at iteration $i$, $\kappa_{xy}^{(i)}$ is the Ollivier Ricci curvature at iteration $i$, and $d_{xy}^{(i)}$ is the distance between $x$ and $y$ at iteration $i$ (itself inferred from the current set of edge weights). A small parameter $\eta$ may be included to control the step size. Some versions set $\eta=1$ (or a small fraction of 1) to maintain stability. Edges that exhibit large positive curvature shrink in length, whereas edges that exhibit large negative curvature expand. Over several iterations, clusters of nodes with strong internal connections typically collapse, whereas the bridging edges between separate clusters get longer.

Mathematically, one can think of this process as a discrete approximation of smoothing out highly curved regions, akin to the continuous case in which narrower “necks” get pinched, dividing the manifold into multiple simpler components. In networks, these necks correspond to edges that connect otherwise densely connected subgraphs. Eventually, to isolate these communities rigorously, one can delete or “cut” edges that have become elongated beyond some threshold.

While the process is not exactly identical to Hamilton’s PDE-based Ricci Flow, it follows the same fundamental principle: curvature is a driver that changes local distances over time. Practical heuristics include stopping after a fixed number of iterations or until some measure of curvature or partitioning becomes stable.

In summary, the Ricci Flow approach in differential geometry has deep and profound implications for how we dissect and understand global properties of curved spaces. By translating it into the language of optimal transport and Ollivier Ricci curvature, we gain a discrete tool that can be applied to purely combinatorial graphs or other abstract metric spaces. This theoretical backdrop provides the motivation for analyzing complex networks---be they social, biological, informational, or otherwise---in a geometric framework, where communities are interpreted as regions of high effective curvature. 

The next section examines how this perspective concretely applies to large-scale networks, and how the \emph{Adjusted Rand Index (ARI)} and \emph{modularity} come into play as metrics to gauge the performance of Ricci Flow-based community detection. We also clarify how the notion of a “cutting point” or “surgery” in the discrete flow context helps isolate the fundamental structural blocks of a network.