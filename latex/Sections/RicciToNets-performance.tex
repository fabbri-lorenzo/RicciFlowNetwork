
\section{Performance considerations}
\label{sec:ricci_flow_networks_performance}

While the continuous Ricci Flow PDE can be computationally demanding in the smooth case, the discrete version has its own set of computational challenges. The major cost typically arises in Wasserstein Distances. Computing $W(m_x, m_y)$ for each edge $(x,y)$. In a graph with $n$ nodes and $e$ edges, if we attempt an exact solution of the optimal transport problem, it can become computationally expensive. However, in many practical discrete settings, especially with local probability measures $m_x$ that concentrate mass on immediate neighbors, $W(m_x,m_y)$ can often be computed with simpler combinatorial formulas or approximations. 

\textbf{Optimal Transportation Distance (OTD) vs.\ Average Transportation Distance (ATD).}
\\
In some formulations, one seeks the exact \emph{optimal transportation distance} (OTD), effectivelysolving the full Kantorovich transport problem between measures $m_x$ and $m_y$. This yields theprecise Wasserstein distance, but can be computationally intensive for large networks. Analternative is to approximate the cost by assigning a simpler, \emph{average} cost of moving massbetween neighbors, often referred to as the \emph{average transportation distance} (ATD). In ATD,one typically does not compute a min-cost matching across all possible edges of transport butinstead uses a simplified procedure (e.g., counting overlaps or partial overlaps of neighborhoods)to estimate the transportation cost. Although ATD may lose some exactness compared to OTD, itgreatly reduces the computational overhead and can still capture important curvature information inlarge-scale networks.
\item \textbf{Repeated Distance Computation:} After each iteration, we update edge weights andpotentially must recompute shortest-path distances $d(x,y)$ for all node pairs or for edges. If $e$is large, repeated all-pairs computations might be costly, unless efficient incrementalshortest-path algorithms or approximate methods are used.


Despite these costs, modern computational resources and heuristics usually make discrete Ricci Flow feasible for networks of moderate size. For very large networks (millions of edges), one might rely on approximations, sampling techniques, or simplified versions of curvature (e.g., proxies to Ollivier Ricci curvature). In particular, using ATD rather than a full OTD calculation can strike a balance between accuracy and efficiency in such large-scale scenarios.



