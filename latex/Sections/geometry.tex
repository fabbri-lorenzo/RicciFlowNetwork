In order to understand the Ricci Flow applications, it's necessary to study the properties of Riemannian Manifolds, in particular how to define a flow on a Manifold and how this is related to some differential equations. We'll see that the Ricci Flow is nothing but a differential equation for some class of different metrics on a Riemannian Manifold.

\section{Differentiable Manfifolds}
Let's first refine the definition of a differentiable Manifold~\ref{def:manifold-prev}.

\begin{definition}[Differentiable manifold]
    $\M$ is an $n$-dimensional differentiable manifold if
    \begin{itemize}
        \item $\M$ is a topological space.
        \item $\M$ is provided with a family of pairs $\{ (U_i, \phi_i) \}$.
        \item $\{U_i\}$ is a family of open sets which covers $\M$, that is, $\cup_i U_i = \M$. $\phi_i$ is a homeomorphism from $U_i$ onto open subsets $U_i'$ of $\R^n$.
        \item Given $U_i$ and $U_j$ such that $U_i \cap U_j \neq \emptyset$, the map $\psi_{ij} = \phi_i \circ \phi_j^{-1}$ from $\phi_j (U_i \cap U_j)$ to $\phi_i(U_i \cap U_j)$ is infinitely differentiable.
    \end{itemize}
\end{definition}

The pair $(U_i, \phi_i)$ is called a \emph{chart}, while the whole family $\{(U_i, \phi_i)\}$ is an \emph{atlas}. The subset $U_i$ is called the \emph{coordinate neighbourhood}, while $\phi_i$ is the \emph{coordinate function}. This homeomorphism is represented by $n$ functions $\{x_1(p), \dots, x_n(p)\}$, which are called \emph{coordinates}. However, notice that a point $p\in\M$ is independent of its coordinates. In each coordinate neighbourhood $U_i$, $\M$ looks like an open subset of $\R_n$ whose element is $\{x^1, \dots x^n\}$, and this explains our previous definition~\ref{def:manifold-prev}.

Recall from the previous section that we're interested in Manifolds with boundaries. We're now able to define them.
\begin{definition}{Manifold with boundary}
    A \emph{Manifold with boundary} is a topological space $\M$ which is covered by a family of open sets $\{U_i\}$, each of which is homeomorphic to an open set of $H^n$, where
    \begin{equation}
        H^n \coloneq \{ (x^1, \dots, x^n) \in R^n | x^n \geq 0 \} .
    \end{equation} 
\end{definition}

The set of points which are mapped to points with $x_n = 0$ is called the \emph{boundary} of $\M$, denoted by $\partial \M$. The coordinates of $\partial \M$ may be given by $n-1$ numbers $(x^1, \dots, x^{n-1},0)$. One must then be careful to define smoothness. Indeed, the map $\psi_{ij} \colon \phi_j (U_i \cap U_j) \to \phi_i(U_i \cap U_j)$ is defined on an open set of $H^n$ in general, and $\phi_{ij}$ is said to be smooth if it's $C^\infty$ in an open set of $\R^n$ which contains $\phi_j(U_i \cap U_j)$.

\section{Differentiable Maps}
Due to the presence of the differentiable structure on a Manifold, we're able to use the calculus techniques developed for $\R^n$. We can define a map between manifolds.

Let $\M$ and $\mathcal{N}$  be an $m$-dimensional and an $n$-dimensional Manifold, respectively. Let's then consider a map between them, i.e., $f: \M \to \mathcal{N}$, where $\M \ni p \mapsto f(p) \in \mathcal{N}$. Taking the charts $(U, \phi)$ on $\M$ and $(V,\psi)$ on $\mathcal{N}$,the coordinate representation of $f$ will be
\begin{equation}
    \phi \circ f \circ \phi^{-1}: \R^m \to \R^n .
\end{equation}
We can write $\phi(p) = \{x^\mu\}$ and $\psi(f(p)) = \{y^\alpha\}$, which makes explicit that $y \equiv \psi \circ f \circ \phi^{-1}(x)$ is an $m$-variables, vector-valued, usual function. Rendering the coordinate charts implicit, we may write $y^\alpha = f^\alpha(x^\mu)$, which makes clear why $f$ is said to be \emph{differentiable} at $p$ if $y$ is $C^\infty$. Notice, however, that the differentiability of $f$ is independent of the chosen coordinates.

A very important type of maps between manifold is given by diffeomorphisms.
\begin{definition}[Diffeomorphism]
    Let $f \colon \M \to \mathcal{N}$ be a homeomorphism and $\psi$ and $\phi$ the same coordinate functions as before. Then, if $\psi \circ f \circ \phi^{-1}$ is invertible and both $y \equiv \psi \circ f \circ \phi^{-1}(x)$ and $x \equiv \phi \circ f^{-1} \circ \psi^{-1}(y)$ are $C^\infty$, $f$ is called a \emph{diffeomorphism} and $\M$ is said to be \emph{diffeomorphic} to $\mathcal{N}$, $\M \equiv \mathcal{N}$.
\end{definition}

Taking a diffeomorphism from a Manifold into itself, we can implement the concept of change of coordinates, in two different interpretation. First, the set of diffeomorphisms $f\colon \M \to \M$ form a group denoted by $\diff(\M)$. Considering a particular $f \in \diff(\M)$, and a chart $(U, \phi)$, such that, for $p \in U$ and $f(p) \in U$, we get $\phi(p) = x^\mu(p)$ and $\phi(f(p))=y^\mu(f(p))$, then $y$ is a differentiable function of $x$ and the above diffeomorphism can be thought as an \emph{active transformation} for a change of coordinates.

However, if $(U,\phi)$ and $(V,\psi)$ are overlapping charts, for a point $p \in U \cap V$, there are two coordinates values, i.e., $x^\mu = \phi(p)$ and $y^\mu = \psi(p)$. Then, the map $x \mapsto y$ is differentiable, and it represents a \emph{passive transformation} for a change of coordinates.

Two important kinds of maps are curves and functions.

\begin{definition}[Curve]
    An \emph{open curve} in an $n$-dimensional Manifold $\M$ is a map $c \colon (a,b) \to \M$, where $(a,b)$ is an open interval such that $a<0<b$. A \emph{closed curve} is a map $c \colon S^1 \to \M$. On a chart $(U,\phi)$, a curve $c(t)$ ha the coordinates representation $x=\phi \circ c \colon \R \to \R^n$.
\end{definition}

\begin{definition}[Function]
    A \emph{function} $f$ on $\M$ is a smooth map from $\M$ to $\R$. On a chart $(U,\phi)$, the coordinate representation of $f$ is given by $f \circ \phi^{-1} \colon \R^n \to \R$, which is a real-valued function of $n$ variables. The set of functions is denotes as $\mathfrak{F}(\M)$.
\end{definition}

\section{Vectors}
The curves allow us to define a vector as a differential operator. So understand this more deeply, let's take a Manifold $\M$, a curve $c\colon (a,b) \to \M$ and a function $f \colon \M \to \R$, where $(a,b)$ is an open interval containing $t=0$, as showed in fig.\color{red}vector\color{black}. The tangent vector at $c(0)$ is defined to be the directional derivative of a function $f(c(t))$ along the curve $c(t)$ at $t=0$. In particular,
\begin{equation}
    X[f] \coloneq \left. \frac{\ud f (c(t))}{\ud t}\right|_{t=0} = \left. \frac{\partial f}{\partial x^\mu} \frac{\ud x^\mu(c(t))}{\ud t} \right|_{t=0} = X^\mu \left(\frac{\de f}{\de x^\mu}\right),
\end{equation}
where we defined
\begin{equation}\label{eq:def-vectors}
    X = X^\mu \left(\frac{\de}{\de x^\mu}\right), \quad X^\mu = \left. \frac{\ud x^\mu(c(t))}{\ud t} \right|_{t=0}
\end{equation}
Notice that we used abuse of notation, that is $\de f / \de x^\mu$ really means $\de (f \circ \phi^{-1}(x))/\de x^\mu$.

Trying to be more precise, as known from multivariate analysis, whenever we have a curve, we can define an equivalence class. This case is no different. Indeed, given two curves $c_1(t)$ and $c_2(t)$ on $\M$, if they satisfy
\begin{subequations}
\begin{gather}
    c_1(0)= c_2(0)=p, \\
    \left. \frac{\ud x^\mu(c_1(t))}{\ud t} \right|_{t=0} = \left. \frac{\ud x^\mu(c_2(t))}{\ud t} \right|_{t=0} ,
\end{gather}
\end{subequations}
then they yield the same differential operator $X$ at $p$. Then, the above conditions define an equivalence relation $\sim$ such that $c_1(t) \sim c_2(t)$. This allows us to define
\begin{definition}[Vector]
    A \emph{tangent vector} $X$ is the \emph{equivalence class of curves}
    \begin{equation}
        [c(t)] = \left\{ \tilde{c}(t) \; \Big| \; \tilde{c}(0) = c(0) \; \land \; \left. \frac{\ud x^\mu (\tilde{c}(t))}{\ud t}\right|_{t=0} = \left. \frac{\ud x^\mu (c(t))}{\ud t}\right|_{t=0} \right\} .
    \end{equation}
\end{definition}
All the tangent vectors at a particular point $p \in \M$ form a vector space called \emph{tangent space} of $\M$ at $p$, denoted by $T_p \M$. By means of equation~\cite{eq:def-vectors}, it's straightforward to see that the basis vectors of this vector space are
\begin{equation}
    .
\end{equation}


\section{One-Forms}
\section{Tensors}
\section{Flow generated by a vector field}