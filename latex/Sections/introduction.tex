%**************** OVERVIEW OF THE PROJECT ******************
The study of networks has gained considerable attention in various fields, ranging from sociology to biology, and beyond. One of the central problems in network science is community detection, where the objective is to identify groups of nodes (communities) that are more densely connected internally than with the rest of the network. Traditional methods for community detection often rely on statistical or combinatorial approaches. However, recent developments in geometric methods have introduced new ways to approach this problem by leveraging concepts from differential geometry \cite{communitydetectionnetworksricci}.

A powerful geometric tool, the Ricci Flow, originally developed in the context of smooth Riemannian manifolds, can be adapted to discrete network structures. In its original formulation, the Ricci Flow evolves the metric of a manifold according to the curvature (represented by Ricci tensor), leading to a smoothing process over time. Ollivier-Ricci curvature, a discretization of Ricci curvature for graphs, provides a framework to extend this idea to networks, where the "curvature" of edges encodes structural information about node connectivity. Specifically, positive curvature tends to shrink intra-community edges, while negative curvature expands inter-community edges \cite{communitydetectionnetworksricci}.

In this project, we explore the application of Ricci Flow in community detection by focusing on planar graphs, i.e. graphs that can be drawn on a plane without any edges crossing each other. By applying Ollivier-Ricci curvature to this network, we aim to assign and iteratively modify the curvature of edges, enhancing intra-community cohesion while stretching inter-community connections. Starting from graphs in which all the edges have weights set to 1, the flow will introduce weight diversity based on the topological characteristics of the graph, allowing observation of the natural curvature-driven "deformation" of the network.

The results will be compared to the predefined community labels, allowing for a direct comparison between the detected communities and the actual community structures. 

The developed code can be accessed in the corresponding GitHub repository: \textit{\href{https://github.com/fabbri-lorenzo/RicciFlowNetwork}{RicciFlowNetwork}}.