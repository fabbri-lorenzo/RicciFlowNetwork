%**************** OVERVIEW OF THE PROJECT ******************
The study of networks has gained considerable attention in various fields, ranging from sociology to biology, and beyond. One of the central problems in network science is community detection, where the objective is to identify groups of nodes (communities) that are more densely connected internally than with the rest of the network. Traditional methods for community detection often rely on statistical or combinatorial approaches. However, recent developments in geometric methods have introduced new ways to approach this problem by leveraging concepts from differential geometry \cite{Ni:communitydetectionnetworksricci}.

A powerful geometric tool, the Ricci Flow, originally developed in the context of smooth Riemannian manifolds, can be adapted to discrete network structures. In its original formulation, the Ricci Flow evolves the metric of a manifold according to the curvature (represented by Ricci tensor), leading to a smoothing process over time. Ollivier-Ricci curvature, a discretization of Ricci curvature for graphs, provides a framework to extend this idea to networks, where the "curvature" of edges encodes structural information about node connectivity. Specifically, positive curvature tends to shrink intra-community edges, while negative curvature expands inter-community edges \cite{Ni:communitydetectionnetworksricci}.

In this project, we apply Ollivier-Ricci curvature and Ricci Flow to detect the two known communities in Zachary’s Karate Club graph. The approach follows the work of Ni et al. \cite{Ni:communitydetectionnetworksricci}, where Ricci Flow is used to reshape edge weights iteratively, enhancing the separation between different communities. After applying Ricci Flow, we perform edge surgery to remove weakly connected edges and extract communities as the connected components of the resulting graph.

The results will be compared to the predefined community labels, allowing for a direct comparison between the detected communities and the actual community structures. 

The developed code can be accessed in the corresponding GitHub repository: \textit{\href{https://github.com/fabbri-lorenzo/RicciFlowNetwork}{RicciFlowNetwork}}.
Code documentation is accessible at \textit{\href{https://fabbri-lorenzo.github.io/RicciFlowNetwork/}{CodeDocumentation}}.

\textcolor{red}{Diciamo che la prima parte del report è introduttiva e riguarda le stringhe ecc.?}