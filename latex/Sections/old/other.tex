Recall from the previous section that we're interested in Manifolds with boundaries. We're now able to define them.
\begin{definition}[Manifold with boundary]
    A \emph{Manifold with boundary} (see fig.~\ref{fig:boundary}) is a topological space $\M$ which is covered by a family of open sets $\{U_i\}$, each of which is homeomorphic to an open set of $H^n$, where
    \begin{equation}
        H^n \coloneq \{ (x^1, \dots, x^n) \in R^n | x^n \geq 0 \} .
    \end{equation} 
\end{definition}

The set of points which are mapped to points with $x_n = 0$ is called the \emph{boundary} of $\M$, denoted by $\partial \M$. The coordinates of $\partial \M$ may be given by $n-1$ numbers $(x^1, \dots, x^{n-1},0)$. One must then be careful to define smoothness. Indeed, the map $\psi_{ij} \colon \phi_j (U_i \cap U_j) \to \phi_i(U_i \cap U_j)$ is defined on an open set of $H^n$ in general, and $\phi_{ij}$ is said to be smooth if it's $C^\infty$ in an open set of $\R^n$ which contains $\phi_j(U_i \cap U_j)$.

Taking a diffeomorphism from a Manifold into itself, we can implement the concept of change of coordinates, in two different interpretation. First, the set of diffeomorphisms $f\colon \M \to \M$ form a group denoted by $\diff(\M)$. Considering a particular $f \in \diff(\M)$, and a chart $(U, \phi)$, such that, for $p \in U$ and $f(p) \in U$, we get $\phi(p) = x^\mu(p)$ and $\phi(f(p))=y^\mu(f(p))$, then $y$ is a differentiable function of $x$ and the above diffeomorphism can be thought as an \emph{active transformation} for a change of coordinates.

However, if $(U,\phi)$ and $(V,\psi)$ are overlapping charts, for a point $p \in U \cap V$, there are two coordinates values, i.e., $x^\mu = \phi(p)$ and $y^\mu = \psi(p)$. Then, the map $x \mapsto y$ is differentiable, and it represents a \emph{passive transformation} for a change of coordinates.




Obviously, $\{e_\mu\}$ are not the only possible basis for the vector space $T_p \M$. Indeed, as known from linear algebra, an arbitrary linear combination $\hat{e}_i \coloneq A_i^\mu e_\mu$, with $A = (A_i^\mu) \in GL(n,\R)$, is a basis as well. In this case, $\{\hat{e}_i\}$ is called a \emph{non-coordinate basis}.

Further, given that a vector exists independently of its coordinates, we can take two coordinate charts, with a point in the intersection of their domains, i.e., $p \in U_i \cap U_j$, with $x = \phi_i(p)$ and $y = \phi_j (p)$. A generic vector $X \in T_p \M$ can be equivalently expanded as
\begin{equation}
	X = X^\mu \frac{\de}{\de x^\mu} = \tilde{X}^\mu \frac{\de}{\de y^\mu},
\end{equation}
which shows the relation of the components of a vector in two different charts, i.e,
\begin{equation}
	\tilde{X}^\mu = X^\nu \frac{\de y^\mu}{\de x^\nu}.
\end{equation}






It's now easy to consider generic one-forms $\omega \in T^*_p \M$ and generic vectors $V \in T_p \M$. Picking a coordinate basis for the tangent space and its dual basis for the cotangent one, we can expand $\omega = \omega_\mu \ud x^\mu$ and $V = V^\mu \de_\mu$. Then, as usual, we can define an \emph{inner product} $\scalar{.}{.}\colon T^*_p \M \times T_p \M \to \R$ by
\begin{equation}\label{eq:def-inner-product-vector-one-form}
	\scalar{\omega}{V} = \omega_\mu V^\mu \scalar{\ud x^\mu}{\de_\nu} = \omega_\mu V^\nu \delta^\mu_\nu = \omega_\mu V^\mu.
\end{equation}

Finally, analogously than before, we can pick two charts and a point in their domain's intersection, $p \in U_i \cap U_j$, so that
\begin{equation}
	\omega = \omega_\mu \ud x^\mu = \tilde{\omega}_\nu \ud y^\nu,
\end{equation}
with $x = \phi_i(p)$ and $y=\phi_j(p)$. Then, we can easily infer the transformation rule of a one-form under a change of coordinates, indeed,
\begin{equation}
	\ud y^\nu = \frac{\de y^\nu}{\de x^\mu} \ud x^\mu \implies \tilde{\omega}_\nu = \omega_\mu \frac{\de x^\mu}{\de y^\nu} .
\end{equation}




Further, taking $r$-vectors $V_a = V_a^\mu \de_\mu$ and $q$ one-forms $\omega_a = \omega_{a\mu} \ud x^\mu$,the action of $T^{(q,r)}$ on them is given by
\begin{equation}
	T(\omega_1, \dots, \omega_q; V_1, \dots, V_r) = \tensor{T}{^{\mu_1}^\dots^{\mu_q}_{\nu_1}_\dots_{\nu_r}} \omega_{1\mu_1} \dots \omega_{q\mu_q} V_1^{\nu_1} \dots V_r^{\nu_r}.
\end{equation}




An important tensor field we're interested in is the metric tensor, which is tightly related to the definition of a Ricci Flow, in the context of Riemannian manifolds. However, let's first study an important property of vector fields: they can generate a flow on a manifold. Studying further the relation among flows and differential equation, allows us to better grasp the definition of a Ricci Flow.







Let $X$ be a vector field in $\M$, $X \in \X(\M)$. An \emph{integral curve} $x(t)$ of $X$ is a curve in $\M$, whose tangent vector at $x(t)$ is $X|_x$. In a chart $(U,\phi)$, this is equivalent to
\begin{equation}\label{eq:ode-integral-curve}
	\frac{\ud x^\mu}{\ud t} = X^\mu(x(t)) ,
\end{equation}
where $x^\mu(t)$ is the $\mu$-th component of $\phi(x(t))$ and $X = X^\mu \de_\mu$. Pay attention to the abuse of notation. Indeed, we used $x$ to denote both a point on $\M$ and its coordinates.

Then, to finding the integral curve of a vector field $X$ is equivalent to solving the system of ODEs~\eqref{eq:ode-integral-curve}, with initial condition $x^\mu_0 = x^\mu (0)$, which are the coordinates of the integral curve at $t=0$.

The existence and uniqueness theorem of ODEs guarantees that there is a unique solution to~\eqref{eq:ode-integral-curve}, at least locally, with the initial condition $x^\mu_0$.

Let, then, $\sigma(t,x_0)$ be an integral curve of $X$ passing through a point $x_0$ at $t=0$ and let's denote its coordinates by $\sigma^\mu(t,x_0)$. Then, eq.~\eqref{eq:ode-integral-curve} becomes
\begin{equation}\label{eq:ode-flow-coordinates}
	\frac{\ud}{\ud t} \sigma^\mu (t,x_0) = X^\mu (\sigma(t,x_0)),
\end{equation}
with the initial condition
\begin{equation}\label{eq:ode-flow-initial-condition}
	\sigma^\mu(0,x_0) = x^\mu_0.
\end{equation}




The purpose of the next chapter will be to rigorously define the Ricci Flow. However, to conclude this motivation part, let's take a brief detour onto string theory, in which, with a simple example in the context of non-linear sigma models, it's possible to grasp the meaning of the Ricci Flow application to understand topological properties of a manifold through renormalization group flows in a Riemannian manifold.

