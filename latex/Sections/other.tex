Recall from the previous section that we're interested in Manifolds with boundaries. We're now able to define them.
\begin{definition}[Manifold with boundary]
    A \emph{Manifold with boundary} (see fig.~\ref{fig:boundary}) is a topological space $\M$ which is covered by a family of open sets $\{U_i\}$, each of which is homeomorphic to an open set of $H^n$, where
    \begin{equation}
        H^n \coloneq \{ (x^1, \dots, x^n) \in R^n | x^n \geq 0 \} .
    \end{equation} 
\end{definition}

The set of points which are mapped to points with $x_n = 0$ is called the \emph{boundary} of $\M$, denoted by $\partial \M$. The coordinates of $\partial \M$ may be given by $n-1$ numbers $(x^1, \dots, x^{n-1},0)$. One must then be careful to define smoothness. Indeed, the map $\psi_{ij} \colon \phi_j (U_i \cap U_j) \to \phi_i(U_i \cap U_j)$ is defined on an open set of $H^n$ in general, and $\phi_{ij}$ is said to be smooth if it's $C^\infty$ in an open set of $\R^n$ which contains $\phi_j(U_i \cap U_j)$.